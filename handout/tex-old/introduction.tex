\documentclass{tufte-handout}

\title{ \sc Introduction to Algorithmic Fairness}


\vspace{5mm}
\author[Marcello Di Bello]{Marcello Di Bello}

\date{} % without \date command, current date is supplied

%\geometry{showframe} % display margins for debugging page layout

\usepackage{graphicx} % allow embedded images
  \setkeys{Gin}{width=\linewidth,totalheight=\textheight,keepaspectratio}
  \graphicspath{{graphics/}} % set of paths to search for images
\usepackage{amsmath}  % extended mathematics
\usepackage{booktabs} % book-quality tables
\usepackage{units}    % non-stacked fractions and better unit spacing
\usepackage{multicol} % multiple column layout facilities
\usepackage{lipsum}   % filler text
\usepackage{fancyvrb} % extended verbatim environments
  \fvset{fontsize=\normalsize}% default font size for fancy-verbatim environments

% Standardize command font styles and environments
\newcommand{\doccmd}[1]{\texttt{\textbackslash#1}}% command name -- adds backslash automatically
\newcommand{\docopt}[1]{\ensuremath{\langle}\textrm{\textit{#1}}\ensuremath{\rangle}}% optional command argument
\newcommand{\docarg}[1]{\textrm{\textit{#1}}}% (required) command argument
\newcommand{\docenv}[1]{\textsf{#1}}% environment name
\newcommand{\docpkg}[1]{\texttt{#1}}% package name
\newcommand{\doccls}[1]{\texttt{#1}}% document class name
\newcommand{\docclsopt}[1]{\texttt{#1}}% document class option name
\newenvironment{docspec}{\begin{quote}\noindent}{\end{quote}}% command specification environment

\begin{document}


\maketitle

\subsection{Different perspectives, disciplines}

\begin{itemize}

 \item[] Technical v. theoretical, conceptual, philosophical v. lived experiences, stories
\item[] Different disciplines: computer science, statistics, philosophy, law, economics,  sociology, history, anthropology

\end{itemize}

\subsection{Algorithms in general}
 
 \begin{itemize}

 \item[] Definition: a series of precisely defined steps that perform a task
   \marginnote{\textsc{Question:} How precisely specified?} 
\item[] General structure: Inputs --> algorithm --> output
\item[] Example: sorting numbers in ascending order
   \marginnote{\textsc{Question:} How would such an algorithm look like?} 
\item[] Tradeoffs: speed v.\ memory 
\item[] Often written by a human, e.g.\ a programmer using Python

\end{itemize}


\subsection{Machine Learning algorithms}

\begin{itemize}

\item[] Meta-algorithms whose input are historical data and whose output is another algorithm.
ML algorithms are \textit{self-programming} 

\item[] Applications: face recognition, translation, prediction


\item[] Example:

	\begin{itemize}
	% \item[] \textit{meta-algorithm}: regression 
	\item[] \textit{input}: historical (training) data about high school GPA and SAT score and college graduation are fed to the meta-algorithm
	\item[]  \textit{model search}: \marginnote{Lines are good for 2-dimensional data (e.g. SAT and GPA) and a binary outcome (graduate/not graduate).}   meta-algorithm searches all possible models, say possibles lines (straight or curved) through the data.
	\item[]  \textit{optimization}: \marginnote{To guard against \textit{overfitting}, split between training  
	and validation data.}	 meta-algorithm selects the model (output algorithm) that perform best, often 
	the model that minimizes errors. 
	\item[] \textit{output}: model (predictive algorithm) for predicting college graduation give high school SAT and GPA 
	\item[] \textit{task}: make a prediction given high school SAT and GPA about college graduation for new individuals not in the historical data
	\end{itemize}

	


\subsection{Error Minimization v. Fairness (According to Computer Scientists)}

\begin{quote}
\dots it may be that the model that minimizes the overall error in predicting collegiate success, when used to make admission decisions, happens 
to falsely reject qualified black applicants more often than qualified white applicants. Why? Because the designer didn't anticipate it. She didn't tell the algorithm  to try to equalize the false rejection rates between the two groups, so it didn't. (p. 10).\end{quote}

\begin{quote}
Writing down precise definitions that capture the essence of critical and very human ideas without becoming overly complex is something 
of an art form, and it is inevitable that in many settings, simplifications---sometimes painful ones---are necessary \dots [This] tension is not an artifact \dots it reflects the inherent difficulty of being precise about concepts that previously have been left vague, such as ``fairness." We believe that the only way to make algorithms better behaved is to begin by specifying what our goals for them might be in the first place (p. 12-13).\footnote{Kearns and Roth (2020), \textit{The Ethical Algorithm}, Oxford University Press}

\end{quote}


\subsection{Allegheny Family Screening Tool (AFST)}


\item[] Outcome variables: proxies for child maltreatment: (a) call and referral to Child and Youth Services (CYS), and (b) child placement in foster care.

\item[] Predictive variables: stepwise probit regression, regression tested 287 variables and eliminated 156, leaving out 131 predictors

\item[] Validation data: Receiver Operating Characteristics (ROC) is 76\%

\begin{quote}
In 2016, therwe 
\end{quote}




\end{itemize}

\end{document}